\documentclass[12pt]{book}
\usepackage[utf8]{inputenc}				% поддержка UTF8
\usepackage[english,russian]{babel}  	% поддержки русского языка
\usepackage{listings}					% листинги с кодом
\usepackage{menukeys}					% кнопки клавиатуры
\usepackage[unicode, pdftex]{hyperref}	% гиперссылки
\usepackage{outlines}					% многоуровневые списки
\usepackage{indentfirst}				% первый абзац
\usepackage{cmap}       				% теперь из pdf можно копипастить русский текст
\usepackage{xcolor}
\usepackage{geometry}
\usepackage{tikz}

\setcounter{secnumdepth}{3}				% вложенность секций до третьего уровня

\usefont{T2A}{ftm}{m}{} % Ужирнение начертания шрифта --- после чего
% выглдяит таймсоподобно и удобнее для чтения
% в плохих условиях.

\geometry{verbose,a4paper,tmargin=2cm,bmargin=2cm,lmargin=2.5cm,rmargin=1.5cm}

\definecolor{codegreen}{rgb}{0,0.6,0}
\definecolor{codegray}{rgb}{0.5,0.5,0.5}
\definecolor{codepurple}{rgb}{0.58,0,0.82}
\definecolor{backcolour}{rgb}{0.95,0.95,0.92}

\lstdefinestyle{mystyle}{
	backgroundcolor=\color{backcolour},   
	commentstyle=\color{codegreen},
	keywordstyle=\color{magenta},
	numberstyle=\tiny\color{codegray},
	stringstyle=\color{codepurple},
	basicstyle=\ttfamily\footnotesize,
	breakatwhitespace=false,         
	breaklines=true,                 
	captionpos=b,                    
	keepspaces=true,                 
	numbers=none,                    
	numbersep=5pt,                  
	showspaces=false,                
	showstringspaces=false,
	showtabs=false,                  
	tabsize=2
}
\lstset{style=mystyle}


\begin{document}
	\tableofcontents
	
	%
% Определение новых команд используемых при написании документа, для единообразия оформления
%

\renewcommand{\cmd}[1]{% команда консоли
	\texttt{#1}
}

\newcommand{\opt}[2]{% опция команды
	\textbf{#1} -- #2
}   

\newcommand{\cfgfile}[1]{% конфигурационный файл или файл с настройками
	\textcolor{red}{#1}
}   

\newcommand{\cfgpath}[1]{% путь к настройкам или части системы
	\textcolor{blue}{#1}
}   

	\chapter{Введение}

Задача данной методички -- дать \textbf{практические навыки} работы в консоли Linux, на примере дистрибутива Debian. В методичке обозначаются основные команды, горячие клавиши, инструменты и варианты их использования, а так же методы быстрой и эффективной работы в консоли Linux. В то же время не уходя слишком далеко в детали, оставляя некоторые инструменты и темы открытыми, чтобы читатель провёл самостоятельное их изучение, посредством чтения оригинальной документации, поиска в \href{https://google.com}{Google}, изучения \href{https://stackoverflow.com}{StackOverflow} и просмотра \href{https://youtube.com}{YouTube}. Последнее менее желательно, т.к. концентрация полезной информации в единицу времени достаточно низкая, а качество и полнота преподносимой информации желает лучшего. Никакой сайт или обучающее видео не смогут сравниться с полнотой оригинальной документации, об этом нужно всегда помнить и акцентировать внимание при изучении \textbf{абсолютно любой} технологии, программы, языка программирования, методологии и т.п.

Умение читать оригинальную документацию на английском языке (по диагонали), при этом находя то что нужно -- является критически необходимым навыком, который нужно развивать для быстрой и эффективной работы, т.к. информации на английском языке всегда было и будет больше, в силу того, что мировое сообщество его выбрало в качестве общего для общения и совместной работой между специалистами разных стран. Так же зачастую информация на английском языке является единственным источником о конфигурировании тех или иных инструментов, поэтому умение читать и понимать технические тексты на английском языке является критически важным навыком для успешной работы в IT-сфере.
\\

\noindent \textbf{P.S.} Для единообразия описания, были приняты следующие обозначения и соглашения:

\begin{itemize}
	\item \keys{ Ctrl + X } -- нажатие сочетания клавиш Ctrl+X
	\item \cmd{pwd} -- команда консоли (программа)
	\item \argum{arg} -- аругмент/параметр команды
	\item \cfgfile{/etc/passwd} -- конфигурационный файл или файл с настройками
	\item \cfgpath{/proc/} -- путь к директории с файлами процессов
\end{itemize}

\noindent  \textbf{P.S.2.} В данном методическом пособии слово \textit{команда} употребляется, как синонимом слова \textit{программа} и по сути таковой и является, если речь не идёт о встроенных командах shell-оболочек таких как: \cmd{sh}, \cmd{bash}, \cmd{zsh}.
	\include{chapter-02}
	\section{man}

\noindent
\cmd{man} - сокращение от manual, команда позволяет просматривать справочную информацию о программах Linux или командах shell-оболочки (\cmd{bash, sh, dash, zsh} и пр.), форматах конфигурационных файлов, специальных файлов устройств, описание системных вызовов или библиотечных вызовов и системных команд администратора. Пример формата вызова \cmd{man}:

\begin{lstlisting}
	
	$ man [section] page
	
\end{lstlisting}	

\begin{itemize}	
	\item \textit{section} тип страницы справочной информации:
	
	\subitem \opt{1}{программы или команды shell-оболочки}
	\subitem \opt{2}{системные вызовы (функции ядра: fork, accept, listen, select, mmap и пр.)}
	\subitem \opt{3}{библиотечные вызовы (функции библиотек: fopen, pow, malloc и пр.)}
	\subitem \opt{4}{специальные файлы (обычно находящиеся в \cfgpath{/dev/}: random, mem, tty и пр.)}
	\subitem \opt{5}{форматы файлов (\cfgfile{/etc/passwd}, \cfgfile{/etc/shadow},  \cfgfile{$\sim$/.ssh/authorized\_keys} и пр.)}
	\subitem \opt{6}{игры}
	\subitem \opt{7}{описания, соглашения и пр.}
	\subitem \opt{8}{команды системного администратора (доступные только для root и/или sudo-пользователя: \cmd{ss, adduser, sysctl} и пр.)}

	\item \textit{page} имя программы, команды, конфигурационного файла, системного вызова и т.д.
\end{itemize}

Посмотреть информацию о команде \cmd{man}:
\begin{lstlisting}
	
	$ man man
	
\end{lstlisting}

После входа в интерактивный режим \cmd{man} доступны следующие функции:

\noindent\keys{ q } -- выход из \cmd{man} (обратить внимание, чтобы раскладка была английской) \\
\keys{ h } -- посмотреть помощь по навигации \cmd{man} \\

\noindent\keys{ u } / \keys{ d } -- пролистать на пол экрана вверх/вниз \\
\keys{ y } или \keys{ \arrowkeyup } -- пролистать на одну строку вверх \\
\keys{ e } или \keys{ \arrowkeydown } -- пролистать на одну строку вниз \\
\keys{ w } / \keys{ PgUp } -- пролистать на один экран вверх\\
\keys{ z } / \keys{ PgDown } -- пролистать на один экран вниз\\
\keys{ g } / \keys{ G }  -- переместиться в начало/конец документа \\

\noindent\keys{ / } + \textit{ввести шаблон поиска} -- прямой поиск по шаблону\\
\keys{ ? } + \textit{ввести шаблон поиска} -- обратный поиск по шаблону\\
\keys{ n } / \keys{ N } -- повторить предыдущий поиск в прямом/обратном направлении\\

Для включения/отключения нумерации строк в режиме просмотра справочной страницы необходимо ввести соответствующие опции и нажать \keys{ Enter }:\\ 
\noindent
\opt{-N}{включить нумерацию строк}\\
\opt{-n}{выключить нумерацию строк}\\

Показать все доступные разделы справочной информации по \cmd{passwd}:
\begin{lstlisting}
	
	$ man -f passwd
	
\end{lstlisting}	


Показать раздел 1 справочной информации для команды \cmd{passwd}:
\begin{lstlisting}
	
	$ man 1 passwd
	
\end{lstlisting}	

Показать раздел 5 справочной информации о формате файла \cfgfile{/etc/passwd}:
\begin{lstlisting}
	
	$ man 5 passwd
	
\end{lstlisting}	

Поиск всех справочных страниц в названии или описании которых встречается \textit{ls}:
\begin{lstlisting}
	
	$ man -k ls
	
\end{lstlisting}	

Поиск по всем справочным страницам сочетания слов \textit{password change}:
\begin{lstlisting}
	
	$ man -K "password change"
	
\end{lstlisting}	
\keys{ Enter } -- открыть страницу из результата поиска\\
\keys{ Ctrl + D } -- пропустить страницу\\
\keys{ Ctrl + C } -- закрыть поиск\\

Просмотреть страницу справки команды \cmd{man} на русском языке (если страница справки с соответствующей русской локалью \textit{ru\_RU} имеется в системе):
\begin{lstlisting}
	
	$ man -L ru_RU man
	
\end{lstlisting}	

\noindent
\cfgfile{/etc/manpath.config} -- файл настройки man-db\\
\cfgpath{/usr/share/man/} -- здесь расположены файлы со справочными страницами\\

	\section{nano}

\noindent
\cmd{nano} -- текстовый редактор. Чтобы создать новый файл с именем \argum{file} или открыть существующий в режиме редактирования, необходимо выполнить команду \footnote{если параметр \argum{file} не указывать, то откроется редактор с пустым документом}:

\begin{lstlisting}
	
	$ nano file
	
\end{lstlisting}	

После открытия редактора \cmd{nano} \ref{fig:nano}, внизу можно увидеть следующие подсказками: \verb|^|\textbf{X} Exit, \verb|^|\textbf{G} Help и т.д. В *nix системах сочетения нажатия клавиши \keys{Ctrl} с нажатием другой клавиши обозначается, как символ \verb|^| и далее название клавиши, так нажатие сочетания клавиш \keys{Ctrl + X} и \keys{Ctrl + C} обозначаются, как \verb|^|\textbf{X} и \verb|^|\textbf{C} соответственно.

\begin{figure}[h]
\centering
	\includegraphics[width=\textwidth]{img/ch1/nano.png}
	\caption{Редактор \cmd{nano} после открытия}
	\label{fig:nano}
\end{figure}

Так же в \cmd{nano} можно открыть более одного файла одновременно, например:
\begin{lstlisting}
	
	$ nano file1 file2 file3
	
\end{lstlisting}	

В результате будут открыты файлы \argum{file1}, \argum{file2} и \argum{file3} в так называемый \textit{буфферах} (экраны, между которыми можно переключаться сочетаниями клавиш \keys{ Atl + > } и \keys{ Atl + < }). Так же можно набор файлов по маске, например:
\begin{lstlisting}
	
	$ nano *.log
	
\end{lstlisting}	

\noindent
\keys{ Ctrl + X } -- закрыть текущий буффер и выйти из \cmd{nano}\\
\keys{ Ctrl + O } -- сохранить изменение текущего буффера в файл\\
\keys{ Ctrl + R } -- открыть файл в текущем буфере\\
(\keys{ Ctrl + R })  + (\keys{ Atl + F }) -- открыть файл в новом буфере\\
(\keys{ Ctrl + R })  + (\keys{ Atl + F }) + (\keys{ Enter }) -- создать новый буффер\\


\subsection*{Редактирование}
\noindent
\keys{ Atl + A }/\keys{ Atl + 6 } -- выделить текст с текущей позиции курсора или отменить выделение\\
\keys{ Ctrl + 6 } -- скопировать выделенный текст\\
\keys{ Ctrl + K } -- вырезать текущую строку или выделенный текст в буфер обмена\\
\keys{ Ctrl + U } -- вставить содержимое буфера обмена в текущую позицию курсора\\
\keys{ Atl + M } -- вырезать от позиции курсора до конца файла\\
\keys{ Ctrl + D } -- удалить символ справа от курсора\\
\keys{ Ctrl + H } -- удалить символ слева от курсора\\
\keys{ Ctrl + M } -- вставить пустую строку\\
\keys{ Atl + R } -- заменить текст или регулярное выражение\\
\keys{ Ctrl + W } -- искать текст или регулярное выражение\\
\keys{ Atl + W } -- повторить последний поиск\\

\subsection*{Навигация}
\noindent
\keys{ PgUp }/\keys{ PgDown } -- пролистать на один экран вверх/вниз\\
\keys{ Ctrl  + }/\keys{ Atl + } -- вперёд/назад  на одно слово\\
\keys{ Atl + E }/\keys{ End } -- в конец текущей строки\\
\keys{ Atl + A }/\keys{ Home } -- в начало текущей строки\\
\keys{ Atl + $\setminus$}/\keys{ / }  -- на первую/последнюю строку файла\\
\keys{ Ctrl + \_ } -- перейти на указанный номер \textbf{строки} или \textbf{строки,позиции}\\

\subsection*{Прочее}
\noindent
\keys{ Atl + C } -- отображение положения курсора включить/отключить\\
\keys{ Ctrl + C } -- показать положение курсора (если в режиме редактирования документа)\\
\keys{ Atl + P } -- отображение пробелов\\
\keys{ Atl + Y } -- подсветка синтаксиса\\
\keys{ Atl + I } -- автоотступы\\
\keys{ Atl + H } -- умная клавиша HOME\\
\keys{ Atl + D } -- подсчитать количество слов, строк и символов в документе\\
\keys{ Atl + M } -- включить/отключить мышку для позиционирования курсора в редакторе\\
\keys{ Atl + N } -- включить/отключить нумерацию строк\\

Открыть файл \argum{file} и перейти к строке \textit{line} и столбцу \textit{col}:
\begin{lstlisting}
	
	$ nano +line,col <file>
	
\end{lstlisting}	

Открыть файл \argum{file} с нумерацией строк:
\begin{lstlisting}
	
	$ nano -l <file>
	
\end{lstlisting}	

Двойное нажатие \keys{Esc}, а затем ввод \textbf{трёхзначного числа} от \textbf{000} до \textbf{255} вставит символ с соответствующим ему hex-кодом.

Открыть файл \argum{file} предварительно сохранив его резервную копию:
\begin{lstlisting}
	
	$ nano -B <file>
	
\end{lstlisting}	

Чтобы указать директорию для сохранения резервных копий редактируемого файла или файлов, используется опция -C:
\begin{lstlisting}
	
	$ nano -BC <dir> <file>
	
\end{lstlisting}	


Открыть файл \argum{file} только для чтения используется ключ -v:
\begin{lstlisting}
	
	$ nano -v <file>
	
\end{lstlisting}	

	
	\appendix
%	\include{appendixA}
	
	
%	\bibliographystyle{plain}
%	\bibliography{links}
	
\end{document}