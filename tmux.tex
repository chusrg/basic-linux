\section{tmux}

\cmd{tmux} -- это терминальный мультиплексор,он позволяет создавать, получать доступ и управлять несколькими терминалами (или окнами). \cmd{tmux} можно отсоединить от экрана и продолжить работу в фоновом режиме, а затем снова подключить \footnote{можно использовать вместо nohup, dtach}

Следующая команда пытается подключиться к существующей сессии (если их несколько \footnote{\textbf{tmux ls} отобразить список всех запущенных сессий}, то к последней в списке запущенных сессий) если она существует, если же сессии нет, то создаётся новая сессия:

\begin{lstlisting}
	
	$ tmux attach || tmux new
	
\end{lstlisting}	


\subsection*{Управление окнами}

После этого вы попадаете в полноценную консоль.

\noindent
\keys{Ctrl+b}+\keys{:} -- режим ввода команд\\
\keys{Ctrl+b}+\keys{s} -- список сессий (находясь в tmux)\\
\keys{Ctrl+b}+\keys{d} -- отключится от текущей сессии (deattach)\\
\keys{Ctrl+b}+\keys{\$} -- переименовать текущую сессию\\

Работа с окнами:

\noindent
\keys{Ctrl+b}+\keys{c} -- создать окно\\
\keys{Ctrl+b}+\keys{w} -- список окон\\
\keys{Ctrl+b}+\keys{,} -- переименовать текущее окно\\
\keys{Ctrl+b}+\keys{0...9} -- перейти в такое-то окно с номером 0...9\\
\keys{Ctrl+b}+\keys{p} -- перейти в предыдущее окно (preview)\\
\keys{Ctrl+b}+\keys{n} -- перейти в следующее окно (next)\\

Работа с панелями:

\noindent
\keys{Ctrl+b}+\textbf{\%} -- разделить текущую панель на две, по вертикали\\
\keys{Ctrl+b}+\textbf{"} -- разделить текущую панель на две, по горизонтали\\
\keys{Ctrl+b}+\keys{→←↑↓} -- перейти на панель, находящуюся в стороне, куда указывает стрелка\\
\keys{Ctrl+b}+\keys{x} -- закрыть панель (или набрать exit в терминале)\\
\keys{Ctrl+b}+\keys{Alt}+стрелочки -- изменение размеров текущей панелей (нажать Ctrl+b, зажать Alt держать, а далее с помощью нажатия стрелочек →←↑↓ установить необходимый размер текущей панели)\\
\keys{Ctrl+b}+\keys{\{} -- перемещать панель по часовой стрелке\\
\keys{Ctrl+b}+\keys{\}} -- перемещать панель против часовой стрелки\\


\subsection*{Команды}

\noindent
\opt{new-session}{создать новую сессию, можно передать имя сессии в опции -s и стартовую директорию в опции -c (new \footnote{\textbf{tmux new} создать новую сессию})}\\
\opt{list-sessions}{вывести список всех запущенных сессий (ls)}\\
\opt{attach-session}{подключиться к уже существующей сессии. В параметре необходимо передать опцию -t и идентификатор сессии (attach\footnote{\textbf{tmux attach} подключение к сессии, либо к единственной, либо последней созданной} \footnote{\textbf{tmux attach -t session1} подключение к сессии session1})}\\
\opt{detach-session}{отключить всех клиентов (или переданного с помощью опции -t) от сессии, переданной в опции -s (detach)}\\
\opt{has-session}{проверить существует ли сессия, аналогично, надо передать идентификатор сессии}\\
\opt{kill-server}{остановить все запущенные сессии}\\
\opt{kill-session}{завершить сессию переданную в параметре с опцией -t \footnote{например tmux kill-session -t session1}}\\
\opt{list-clients}{посмотреть клиентов, подключенных к сессии с опцией -t}\\
\opt{list-commands}{список поддерживаемых комманд}\\

\subsection*{Копирование и вставка}

Одна из достаточно важных операций при работе с терминалом - это возможность что-то скопировать и куда-то перенести. После активации поддержки мышки вы можете просто выделить участок текста мышкой и он автоматически скопируется во внутренний буфер tmux

\keys{Ctrl+B}+ \textbf{[} -- переход в режим копирования \footnote{для выхода из режима копирования используйте клавиши q или Esc}, после перехода в режим копирования можно перемещать курсор к нужному месту с помощью стрелок \footnote{этот режим также можно использовать также для прокрутки}\\
\keys{Ctrl+пробел} -- начать выделение области\\
\keys{Ctrl+w} -- копировать выделенную область\\
\keys{Ctrl+b}+\textbf{]} -- вставить текст из внутреннего буфера обмена \\



