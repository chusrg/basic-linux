\section{vim}

\noindent
\cmd{vim} -- самый популярный текстовых редактор предустановленный на всех *nix-like системах, поэтому умение работать в нём является \textbf{обязательным}. Для ознакомления с базовой работой в \cmd{vim} можно воспользоваться встроеным мини-учебником \cmd{vimtutor}, который можно вызвать и командной строки. Чтобы запустить его русскую версию необходимо выполнить команду:

\begin{lstlisting}
	
	$ vimtutor ru
	
\end{lstlisting}	

В редакторе \cmd{vim} есть два режима работы:
\begin{itemize}
	\item \textbf{командный режим} -- чтобы в него перейnи необходимо нажать клавишу \keys{Esc}
	\item \textbf{режим ввода текста} -- чтобы в него перейти нужно нажать клавишу \keys{i}
\end{itemize}


\noindent
Создать или открыть файл с именем \argum{file} для редактирования \footnote{если параметр \argum{file} не указывать, то откроется редактор с пустым документом, сохранить который в дальнейшем возможно посредством ввода команды \textbf{:w} \argum{file} + \keys{Enter}}:

\begin{lstlisting}
	
	$ vi file
	
\end{lstlisting}	

После запуска \cmd{vim} редактор по умолчанию находится в \textbf{командном режиме} и в дальнейшем вся работа в редакторе строится посредством переключения между этими двумя режимами по мере необходимости и работе в каждом из них. В \textbf{командном режиме} вводить текст невозможно, но с помощью нажатия ряда клавиш (комбинации которых будут описаны ниже), возможно редактирование документа и навигация по нему. Стоит отметить, что в \textbf{командном режиме} при вводе команд они \textbf{не отображаются} на экране, за исключением команд, которые начинаются со специальных символов <<\textbf{:}>>, <<\textbf{/}>> или <<\textbf{?}>>. Если в же в \textbf{командном режиме} вводится команда и она начинается с символов <<\textbf{:}>>, <<\textbf{/}>> или <<\textbf{?}>>, в этом случае они отображаются в самой нижней строке экрана, она никак не связана с редактируемым документом и является \textbf{командной строкой} редактора \cmd{vim}. 

Если же необходимо вводить новые символы в документе или удалить посимвольно ранее набранный текст, то необходимо переключиться в \textbf{режим ввода текста} посредством нажатия клавиши \keys{i}, для всех остальных манипуляций с текстом в \cmd{vim} используется \textbf{командный режим}, в который можно перейти посредством нажатия клавиши \keys{Esc}.

\noindent
Открыть файл \argum{file} для просмотра:
\begin{lstlisting}
	
	$ view file
	
\end{lstlisting}	

\noindent
Открыть последнюю сохраненную версию файла \argum{file} после аварийного выхода:
\begin{lstlisting}

	$ vi -r file
	
\end{lstlisting}	

\noindent
Открыть файл \argum{file} и поместить курсор на последнюю строку \argum{n}:
\begin{lstlisting}
	
	$ vi +n file
	
\end{lstlisting}	

\noindent
Открыть файл \argum{file} и поместить курсор на последнюю строку:
\begin{lstlisting}
	
	$ vi + file
	
\end{lstlisting}	

\noindent
Открыть файл \argum{file} и поместить курсор на первое вхождение \argum{string}:
\begin{lstlisting}
	
	$ vi +/string file
	
\end{lstlisting}	

\noindent
Чтобы создать или открыть файл \argum{file} с использованием шифрования, необходимо выполнить команду, а далее ввести ключ для шифрования/дешифрования в зависимости от того создаётся ли файл или открывается уже шифрованный:

\begin{lstlisting}
	
	$ vi -x file
	
\end{lstlisting}	



\noindent
\keys{:} -- переход в режим \textbf{командной строки} \footnote{как таковой клавиши \keys{:} нет, для получения данного символа необходимо нажать сочетание \keys{Shift + ;} в английской расскладке клавиатуры, необходимо обратить на этот факт особое внимание, т.к. в дальнейшем, например, при указании например клавиши \keys{\$} подразумевается нажатие сочетания \keys{Shift + 4}, а при указании клавиши \keys{\_} подразумевается нажатие сочетания клавиш \keys{Shift + -} и т.д.}, не путать с \textbf{командным режимом}. Суть режима заключается в том, что после введения двоеточия, будучи в \textbf{командном режиме}, далее появляется возможность вводить ряд команд с целью управления документом и состоянием редактора в появившейся снизу командной строке, а так же навигации по списку введённых команд в \textbf{командной строке} в \cmd{vim}, посредством стрелок на клавиатуре \keys{ \arrowkeyup }/\keys{ \arrowkeydown }. Пример таких команд перечислен ниже:

\noindent
\opt{:q}{выход из текстового редактора \footnote{фактически это нажатие сочетания клавиш \keys{Shift+;} и далее нажатие \keys{q} и для выполнения команды нажать \keys{Enter}. Если набираемая команда выводится не снизу экрана в \textbf{командной строке} \cmd{vim}, то скорее всего текущий режим это \textbf{режим ввода текста}, необходимо перейти в режим \textbf{командной строки} посредстом нажатия \keys{Esc} и повторить ввод команды}}\\
\opt{:q!}{выход без сохранения}\\
\opt{:wq}{выход с сохранением}\\
\opt{:w}{сохранить файл или внесенные изменения}\\
\opt{:e}{загрузить файл для редактирования \footnote{после ввода команды ставится пробел и вводится имя файла, которое требуется закрузить или вводятся первые буквы названия файла и для \textbf{автодополнения} жмётся кнопка \keys{Tab} (иногда несколько раз, если есть несколько файлов, которые начинаются с указанных первых букв), пока не дополнит до искомого имени файла. Это всё работает, если файл находится в той же директории, в противном случае указывается путь к файлу и только в конце указывается требуемый файл. В процессе указания полного пути к файлу рекомендуется пользоваться автодополнением с помощью нажатия клавиши \keys{Tab}}}\\

\subsection*{Редактирование}
В \textbf{командном режиме} действуют следующие комбинации клавиш для редактирования текста:

\noindent
\opt{yy}{скопировать строку в буфер обмена (нажатие \keys{y + y})}\\
\opt{dd}{вырезать строку в буфер обмена (нажатие \keys{p + p})}\\
\opt{p}{вставить текст из буфера обмена (нажатие \keys{p})}\\

\noindent
\opt{u}{отменить последнюю команду}\\
\opt{U}{отменить все изменения в строке}\\
\keys{Ctrl+r}{отмена отмены последне команды}\\
\opt{.}{позволяет повторить последнее действие}\\

\noindent
\opt{o}{вставить пустую строку строчкой выше курсора и перейти в режим ввода текста}\\
\opt{O}{вставить пустую строку строчкой выше курсора и перейти в режим ввода текста}\\
\opt{х}{удаление одного символа под курсором \footnote{x -- похожа на крестик, удалить/<<перечеркнуть>>}}\\
\opt{X}{удаление одного символа до курсора}\\
\opt{r}{единичная замена символа под курсором \footnote{r -- от слова \textbf{r}eplace}}\\
\keys{Shift+r} – режим замены символов\\

\noindent
\opt{\~}{смена регистра символа над курсором }\\
\opt{J}{слияние следующей строки с текущей (j -- от слова \textbf{j}oin)}\\
\opt{nJ}{слияние \textbf{n} строк (вводится с клавиатуры число \textbf{n} и далее жмётся кнопка \keys{J})}\\

В \textbf{режиме ввода текста} действуют некоторые комбинации клавиш для упрощения работы:

\noindent
\keys{Ctrl+h} -- удаляет символ слева от курсора, аналогично клавише \keys{\arrowkeyleft Backspace}\\
\keys{Ctrl+w} -- удаляет одно слово перед курсором\\
\keys{Ctrl+u} -- удаляет все символы от начала строки до курсора\\
\keys{Ctrl+t} -- вставить табуляцию в начало текущей строки\\
\keys{Ctrl+d} -- удалить табуляцию из начала текущей строки\\

\subsection*{Выделение}
\noindent
\keys{V} -- выделить текст построчно\\
\keys{v} -- выделить кусок текста \footnote{далее возможно использовать команды \textbf{d} -- вырезать выделенный текст в буфер обмена, \textbf{y} -- скопировать выделенный текст в буфер обмена с дальнейшей вставкой посредством команды \textbf{p}}\\
\keys{o}/\keys{O} — перемещают курсор в разные концы выделенного блока для изменения размеров\\
\keys{Ctrl+v} -- выделить прямоугольную часть текста\\

\subsection*{Навигация}

\noindent
\keys{h}/\keys{l} -- переместить курсор на один символ влево/вправо\\
\keys{j}/\keys{k} -- переместить курсор на одну строку вниз/вверх\\
\keys{ \arrowkeyup }/\keys{ \arrowkeydown }/\keys{ \arrowkeyleft }/\keys{ \arrowkeyright } -- так же возможна навигация\\

\noindent
\opt{:number}{перейти на строку с номером \textit{number}}\\
\opt{<number>G}{перейти на конкретную строку с номером \textit{<number>}}\\
\opt{<number>gg}{перейти на конкретную строку с номером \textit{<number>}}\\

\noindent
\keys{z.} -- сделать текущую строку средней строкой экрана\\
\keys{z-} -- сделать текущую строку нижней строкой экрана\\

\noindent
\keys{0}/\keys{\$} -- переместить курсор в начало/конец строки\\
\keys{gg}/\keys{G} -- переместить курсор в начало конец файла\\
\keys{Ctrl+D}/\keys{Ctrl+U} -- на пол экрана вниз/вверх\\
\keys{\}}/\keys{\{} -- абзац вниз/вверх\\

\noindent
\keys{w} -- перейти на начало следующего слова\\
\keys{e} -- перейти к концу следующего слова\\
\keys{b} -- перейти на начало предыдущего слова\\

Также для удобной навигации, в тексте можно расставлять свои метки/<<закладки>> -- это специальный образом отмеченные позиции, куда можно в любой момент вернуть каретку курсора, набрав соответствующую команду. Именем метки может быть любая \textbf{одна буква}. На примере метки с именем <<a>> посмотрим как это работает:

\noindent
\opt{ma}{создание метки}\\
\opt{'a}{перемещение курсора на метку <<a>>}\\
\keys{Ctrl+o}/\keys{Ctrl+i} — перемещение к ранее созданным меткам назад/вперед\\
\opt{:marks}{показать все созданные метки}\\

\subsection*{Составление команд}
Перед большинством команд, начинающихся с двоеточия, может быть указан диапазон строк, на которые эта команда будет действовать. Например, \textbf{:3,7d} служит для удаления строк 3-7. Диапазоны обычно используются с командой :s для замены в нескольких строках, например \textbf{:.,\$s/pattern/string/g} выполнит замены с текущей строки до конца файла.

\noindent
\opt{:n,m}{строки с \textbf{n} до \textbf{m}}\\
\opt{:.}{текущая строка}\\
\opt{:\$}{последняя строка}\\
\opt{:'c}{строка с маркером \textbf{c}}\\
\opt{:\%}{все строки файла}\\
\opt{:g/pattern/}{все строки, содержащие \textbf{pattern}}\\


Одним из преимуществ \cmd{vim} перед рядом других редакторов, является возможность составлять комбинации из команд, например:

\noindent
\opt{4dd}{вырезать четыре строки}\\
\opt{3e}{перейти на три слова вперёд}\\
\opt{7x}{удалить 7 символов}\\
\opt{8xj}{заменить следующие 8 символов на символ <<j>>}\\
и т.д.

\subsection*{Поиск и замена}

Команды для поиска:

\noindent
\keys{/} -- прямой поиск текста\\
\keys{?} -- обратный поиск текста\\
\keys{n} -- повторить поиск вперёд\\
\keys{N} -- повторить поиск назад\\

Команды для поиска и замены:

\noindent
\opt{:s/old/new}{заменить первое вхождение \textbf{old} на \textbf{new}}\\
\opt{:s/old/new/g}{заменить все вхождения \textbf{old} на \textbf{new} во всём файле}\\
\opt{:\%s/old/new/g}{замена всех вхождений \textbf{old} на \textbf{new} во всём файле}\\
\opt{:\%s/old/new/gc}{замена всех вхождений \textbf{old} на \textbf{new} во всём файле с запросом подтверждения каждой замены (<<c>> — confirmation)}\\
\opt{:N,Ms/old/new/g}{заменить вхождение \textbf{old} на \textbf{new} в диапазоне строк от \textbf{N} до \textbf{M}}\\

\subsection*{Макросы}

Макросы последовательность команд выполненных в \cmd{vim}, которая записана в именованный \textit{регистр}, который в дальнейшем можно вызвать более лаконично, введя в командном режиме @ и имя регистра (один символ). \textit{Регистры} обозначаются латинскими буквами без учёта регистра (<<a>> и <<A>> -- один и тот же регистр), цифрами, и даже специальными символами. Пример записи макроса с именем <<a>>:
\begin{itemize}
\item \textbf{qa} -- начало записи макроса \textbf{@a}
\item ...
\item набор команд \cmd{vim}
\item ...
\item \textbf{q} -- окончание записи макроса
\end{itemize}

Теперь, чтобы запустить макрос с именем <<a>>, нам нужно ввести команду \textbf{@a}, и запустится на выполнение макрос, хранящийся в регистре \textbf{a}. Мы можем традиционно для \cmd{vim} написать команду \textbf{10@a}, и макрос будет выполнен 10 раз.

\subsection*{Буфера}

\noindent
Открыть файлы \argum{file1}, \argum{file2} и \argum{file3}:
\begin{lstlisting}
	
	$ vi file1 file2 file3
	
\end{lstlisting}	

После открытия набора файлов их загрузка происходит в \textbf{буфера}, а далее происходит отображение содержимого буфера в \textbf{окнах} и \textbf{вкладках} о которых будет рассказано ниже.

\noindent
\opt{:ls}{список открытых буферов}\\
\opt{:bn}{перейти к следующему буферу}\\
\opt{:bp}{перейти к предыдущему буферу}\\

\noindent
\opt{:b name}{переключиться на буфер \textbf{name} \footnote{(очень удобно комбинируется с табом, к примеру пишем \textbf{:b} + пробел, нажимаем несколько раз \keys{Tab} с помощью автоподстановки меняются имена открытых буферов}}\\
\opt{:bd}{удалить текущий буфер (если этот буфер единственное окно то \cmd{vim} закроется)}\\
\opt{:bd name}{удалить буфер \textbf{name}}\\
%%%%%%%%%%%%%%%%%%%%%%%%%%%%%%%%%%%%%%%%%%%%%%%%%%%%%%%%%%%%

\subsection*{Окна}

Открыть в \textbf{vim} каждый файл из строки аргументов в отдельном окне \textbf{-o} \footnote{ключ \textbf{-O} приводит к аналогичным результатам, но окна будут разделены по вертикали}:

\begin{lstlisting}
	
	$ vi -o one.txt two.txt three.txt
	
\end{lstlisting}	

\noindent
\opt{:new}{создать окно}\\
\opt{:sp}{разделить окно по горизонтали (\keys{Ctrl+w}+\keys{s})}\\
\opt{:vsp}{разделить окно по вертикали (\keys{Ctrl+w}+\keys{v})}\\
\opt{:only}{развернуть текущее окно, закрыв все остальные окна (\keys{Ctrl+w}+\keys{o}). Однако, если в одном из окон есть несохранённые изменения, то вы увидите сообщение об ошибке и это окно не будет закрыто}\\
\opt{:close}{закрыть текущее окно}\\
\keys{Ctrl+w}+<<стрелочки>>(\keys{ \arrowkeyup }/\keys{ \arrowkeydown }/\keys{ \arrowkeyleft }/\keys{ \arrowkeyright }) -- перемещение между окнами\\

Команды для установки и изменения размера окон:

\noindent
\keys{Ctrl+w}+\keys{\_} -- развернуть окно по вертикали\\
\keys{Ctrl+w}+\keys{|} -- развернуть окно по горизонтали\\
\textbf{height}+\keys{Ctrl+w}+\keys{\_} -- установить высоту окна равную \textbf{height}\\
\textbf{width}+\keys{Ctrl+w}+\keys{|} -- установить ширину окна равную \textbf{width}\\
\keys{Ctrl+w}+\keys{=} -- сбросить размер всех разделенных окон и сделать их одинаковыми\\

\noindent
\textbf{count}+\keys{Ctrl+w}+\keys{+} -- увеличение размера окна по вертикали\footnote{если перед командами параметр \textbf{count} не вводится с клавиатуры, то по умолчанию он равен единице}\\
\textbf{count}+\keys{Ctrl+w}+\keys{-} -- уменьшение размера окна по вертикали\\
\textbf{count}+\keys{Ctrl+w}+\keys{>} -- увеличение размера окна по горизонтали\\
\textbf{count}+\keys{Ctrl+w}+\keys{<} -- уменьшение размера окна по горизонтали\\


Команды для перемещения положения окон:

\noindent
\keys{Ctrl+w}+\keys{K} -- поместить текущее окно в верхней части экрана \footnote{\keys{K} это нажатие сочетания \keys{Shift+k}}\\
\keys{Ctrl+w}+\keys{H} -- поместить текущее окно в левой части экрана\\
\keys{Ctrl+w}+\keys{J} -- поместить текущее окно в нижней части экрана\\
\keys{Ctrl+w}+\keys{L} -- поместить текущее окно в правой части экрана\\

Если открыто несколько окон и нужно произвести манипуляции применительно к ряду окон, чтобы не вводить команды в каждом из окон, пригодятся следующие команды:

\noindent
\opt{:qall}{выйти изо всех окон}\\
\opt{:qall!}{выйти изо всех окон, не обращая внимание на несохранённые изменения}\\
\opt{:wall}{сохранить изменения во всех окнах}\\
\opt{:wqall}{сохранить изменения во всех окнах, где это требуется и затем выйти изо всех окон}\\

\subsection*{Вкладки}

Открыть в \textbf{vim} каждый файл из строки аргументов в отдельной вкладке:
\begin{lstlisting}
	
	$ vi -p one.txt two.txt three.txt
	
\end{lstlisting}	

\noindent
\opt{:tabs}{список всех открытых вкладок}\\
\opt{:tabm 0}{открыть первую вкладку}\\
\opt{:tabm}{открыть последнюю вкладку}\\
\opt{:tabn}{перейти в следующую вкладку(\keys{g+t})}\\
\opt{:tabp}{перейти в предыдущую вкладку(\keys{g+T})}\\
\opt{:tabnew file}{открыть файл \textbf{file} в новой вкладке}\\
\opt{:tabedit file}{открыть файл \textbf{file} в новой вкладке}\\
\opt{:tabonly}{закрыть все вкладки кроме текущей}\\
\opt{:tabdo command}{выполнить команду \textbf{command} над содержимым всех вкладок}\\
\opt{:tabclose}{закрыть текущую вкладку}\\
\opt{:tabclose i}{закрыть i-ю вкладку}\\

\begin{itemize}
	\item \textbf{буферы} -- это некие контейнеры, которые содержат в себя загруженные в vim файлы
	\item \textbf{окна} -- это часть экрана монитора, в котором отражается содержимое буфера
	\item \textbf{вкладки} -- это способ организации нескольких окон на экране
\end{itemize}

Переключаясь между вкладками, вы переходите от одного набора окон - к другому набору окон. При этом содержимое любого из буферов вы можете отражать в том окне или в ином окне, как захотите. Таким образом, вы можете увидеть какой-то файл на одной вкладке, отредактировать его, потом переключиться на другую вкладку и увидеть в другом окне тот же самый файл со всеми последними изменениями.


\subsection*{vimdiff}

Чтобы сравнить два файла между собой удобно пользоваться командой \cmd{vimdiff}: 
\begin{lstlisting}
	
	$ vimdiff one.txt two.txt
	
\end{lstlisting}	

\noindent
\opt{]C}{перейти на следующее место в файле, в котором есть отличия}\\
\opt{[C}{перейти на предыдущее место в файле, в котором есть отличия}\\
\opt{do}{diff obtain, получить текст из другого окна, которого не хватает в текущем}\\
\opt{dp}{diff put, записать текст из текущего окна в другое}\\

Строки, которые повторяются в обоих файлах, сложены в одну строку, которая называется \textbf{складкой} и начинается с символов <<+-->>. Наведя курсор на \textbf{складку}, её можно раскрывать и сворачивать используя следующие команды:

\noindent
\opt{zo}{open, раскрывает складку}\\
\opt{zc}{collapse, сворачивает складку}\\

\subsection*{Другие команды}

\noindent
\opt{:set mouse=a}{включить возможность позиционирования курсора в редакторе мышкой}\\
\opt{:set nu}{включить нумерацию строк (или \textbf{:set number})}\\
\opt{:set nonu}{выключить нумерацию строк (или \textbf{:set nonumber})}\\
\opt{:set tabstop=4}{установить размер табуляции равный 4 символам}\\
\opt{:!command}{выполнить команду UNIX не покидая редактора}\\
\opt{:set nolist}{отключить отображение скрытых символов}\\
\opt{:set list}{включить отображение скрытых символов \footnote{спецсимволы в текущей строке: tab (\^ l), backslash, backspace (\^ H), newline (\$), bell (\^ G), formfeed (\^ L\^)}}\\
\opt{:.=}{номер текущей строки}\\
\opt{:=}{количество строк в файле}\\
\keys{Ctrl+G}{имя файла, номер строки, общее число строк и положение в файле}\\
\opt{:r file}{вставить содержимое file после текущей строки}\\
\opt{:nr file}{вставить содержимое file после строки \textbf{n}}\\
:\opt{:history}{показать историю вводимых команд}\\
:\opt{:set tabpagemax=15}{установить максимальное количество вкладок равное 15}\\



\subsection*{Переключение между консолью bash и редактором vi}

В процессе работы с \cmd{vim} может появиться необходимость переключиться в командную оболочку \cmd{bash} и провести ряд манипуляций, для этого случая следует нажать сочетание клавиш \keys{Ctrl+z}, тогда текущий экземпляр \cmd{vim} приостановится и перейдёт в список фоновых заданий. Чтобы обратно вернуться в \cmd{vim}, после необходимых манипуляций в командной оболочке \cmd{bash} необходимо посмотреть список всех фоновых заданий с помощью следующей команды:

\begin{lstlisting}
	
	$ jobs
	
\end{lstlisting}	

И далее запустить соответствующую задачу по номеру её идентификатора \textit{id} \footnote{в нашем случае с \textit{id} равным 1} с помощью команды:
\begin{lstlisting}
	
	$ fg %1
	
\end{lstlisting}	

При управлении фоновыми заданиями в командной оболочке \cmd{bash} доступны следующие команды:

\noindent
\opt{jobs} -- список всех фоновых задач с их \textit{id}\\
\opt{bg \%n} -- поместить задачу с \textit{id} равным \textbf{n} на задний план\\
\opt{fg \%n} -- вернуть задачу с \textit{id} равным \textbf{n} из заднего плана на передний план\\
\opt{kill \%n}{для прерывания фоновой задачи с \textit{id} равным \textbf{n}}\\
\keys{Ctrl+z} -- остановить текущую задачу и поместить её на задний план

Чтобы запустить сразу задачу в фоне из командной строки нужно добавить после команды символ \textbf{\&}, например:
\begin{lstlisting}
	
	$ sleep 3 &
	
\end{lstlisting}	

Но для переключения между окнами лучше всего использовать терминальный мультплексор \cmd{tmux}, о котором более подробно будет рассказано в следующей главе.

\subsection*{Плагины}
% TODO: описать утсановку плагинов и часто используемые плагины в vim'е